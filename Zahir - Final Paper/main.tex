\documentclass[10pt,twocolumn]{article}

% use the oxycomps style file
\usepackage{oxycomps}
\usepackage{xcolor}
\addbibresource{ref.bib}


\title{OxyGPT - A large language model for Occidental College}
\author{Zahir Choudhry}
\affiliation{Occidental College}
\email{zchoudhry@oxy.edu}
\date{May 2024}

\begin{document}

\maketitle

\section{Abstract}
The purpose of this project is to create a centralized source of Occidental College data that relays accurate information pertaining to the school's resources. In creating a retrieval augmented generation tool, I look to provide parties interested in the college an alternative to navigating the complex website provided by Occidental. The system I have developed involves creating a custom web scraper / web crawler program built on pre-existing web scraping libraries to collect data related to Occidental College and then leveraging a vector database as well as natural language processing techniques to provide context to a large language models that will then generate appropriate answers to all questions that are related to Occidental College. Preliminary results show a positive ability to answer basic and medium complexity questions, with harder questions getting positive but mixed results. This approach demonstrates the potential of large language models to improve accessibility and overall enhance the user experience for resource rich websites. 

\section{Problem Statement}
The website for Occidental College presents a challenge to users seeking to find specific resources. While full of valuable information, the dense structure of its web pages makes finding relevant details a frustrating experience, whether it be finding financial aid resources or simply checking the cafeteria menu. My project aims to address this issue by creating a retrieval augmented generation (RAG) tool that integrates the college’s existing online resources into a centralized, large language model (LLM) driven system, enabling users to access their desired content in one convenient location. Accounting for the current limitations \cite{lakera_hallucinations_llms} in large language models such as accuracy issues and hallucinations, this project prioritizes accuracy of its information over retrieval speed to ensure the user can rely on the information generated while keeping the tool accessible.
\vspace{20pt}
\section{Technical Background}
This section will serve two purposes: 1. To introduce anyone unfamiliar with ‘RAG’ and ‘LLM’s to basic explanations of relevant processes and systems. 2. To provide more in depth explanations of processes I took for my project without disrupting the flow of my other sections. So if you see a process that you are confused about, try checking here for more information and clarification.

\subsection{Large Language Model (LLM)\cite{IBM_LLM}}A type of machine learning model trained on an immense amount of data, making the model capable of understanding and generating natural language capable of performing a wide variety of tasks. Popular examples of LLMs include ChatGPT, Anthropic’s Claude, Cohere, and Meta AI. Here is the system prompt that I fed all of my LLMs:  "You are a highly knowledgeable assistant with access to a range of documents about Occidental College. Use the following documents to answer the question accurately: \emph{This is where the context would be inserted}.Answer the following question based on the context provided"

\vspace{10pt}

\subsection{Retrieval Augmented Generation (RAG) \cite{Google_RAG}} An AI framework that combines traditional databases with large language models. It involves taking a query from an LLM and using it to search the separate database. Using the files found querying the database, those files are combined with the original query and sent to an LLM to generate a response. For my project, I fetched 3 files to provide context to the LLM. 
\vspace{10pt}
\subsection{Vector Database \cite{pinecone2024}} A specialized system designed to store and manage high-dimensional numerical vectors. Vector databases use algorithms like the Approximate Nearest Neighbor (ANN) search and Cosine Similarity Search to quickly retrieve semantically similar data for a given vector. 
\vspace{60pt}
\subsection {Vector Embeddings \cite{pinecone_embeddings}} A dense, high dimensional vector that represents data, in the case of this project text but can also represent images and audio. These embeddings are capable of capturing semantic information from data which enables machine learning models to process and understand complex information. In the case of my project, I experimented with embeddings with 384 and 768 dimensions, but my final product used 768 dimensions only.
\vspace{10pt}
\subsection{Chunking and Overlap \cite{langchain_long_text}} The process of dividing extensive / lengthy text into smaller ‘chunks’. Chunking ensures that information is fits within the token limit of a given sentence transformer as well as for large language models, because LLMs have token limits. Overlap is the repeated inclusion of portions of text across different chunks. Overlap is important for maintaining context across chunks. In this project, I set my chunk size to 500 characters and the overlap to 20 characters
\vspace{10pt}
\subsection {Token \cite{langchain_tokens} } The fundamental unit of text that a language model processes. Tokens can range from whole words to characters. This allows for language models to handle text in an effective and efficient manner, especially when the words are complex. An example of different types of tokenization for “Fantastic”. 
1. Whole word: “Fantastic”
2. Partial (3 characters): “Fan”, “tas”, “tic”
3. Character: “F”, “a”, “n”, “t”, “a”, “s”, “t”, “i”, “c”
\vspace{10pt}
\subsection{Sentence Transformer \cite{huggingface_train_sentence_transformers} } A type of machine learning model designed to convert sentences and paragraphs into fixed-size dense vector representations, also known as vector embeddings. This project utilized the ‘all-mpnet-base-v2’ sentence transformer provided by the Huggingface online library
\vspace{10pt}
\subsection {Web Scraping \cite{wikipedia_web_scraping} } The automated process of extracting data from websites. This is a widely used technique for data collection and analysis. BeautifulSoup4 was the webscraping library used in this project.
\vspace{10pt}
\subsection {Web Crawling \cite{botscraper_web_crawling}} a web crawler (also known as a ‘spider’) is a bot that systematically browses through the internet or in this case a set of curated webpages. The spider will travel from page to page and log data that will later be analyzed. Scrapy was the webcrawling library used in this project.
\vspace{10pt}
\subsection {Similarity Search \cite{mljourney_cosine_euclidean}} The process of retrieving vectors that closely match the query vector. This process is critical for applications that depend on being able to identify items with similar characteristics. The two primary metrics for assessing similarity are Euclidean Distance and Cosine Similarity. These metrics will be discussed further in the methods section where I will weight their pros and cons and break down how I decided which method to use.

\section{Prior Work}
This section will synthesize the influence that led to the creation of this project along with any relevant research that aided me in the choices I made when making my project.
\subsection{ChatGPT: LLMs}
My main influence for creating this project was OpenAI’s development of ChatGPT and advancements since its initial release in 2022. Baharini et al. paper “ChatGPT: Applications, Opportunities, and Threats” \cite{arxivChatGPTApplications} details how ChatGPT uses its transformer architecture with pre-trained deep learning models to generate natural language conversations. The paper also takes into account potential ethical concerns, hallucinations,and accuracy issues that led me to focus on accuracy over response time in my project. This paper served as the core inspiration of my comps: “Use large language models to improve the everyday life of people at Occidental College”.
\subsection{Training a LLM}
During the initial research for my project, I investigated how to train and fine-tune large language models to understand their development. A key resource to learning about this process was "A Comprehensive Overview of Large Language Models" by Humza Naveed et al.\cite{naveed2024comprehensiveoverviewlargelanguage}, which explains the steps of training LLMs, including the use of large datasets and the preprocessing required to clean and tokenize the data for effective training. The paper also highlights the two-stage process of pre-training on a vast corpus of data to establish foundational knowledge, followed by fine-tuning on specific tasks to enhance task-specific performance. This deeper understanding of how LLMs are trained showed me one of the fundamental ways to approach my project and a potential pathway to consider. These insights influenced my approach to leveraging pre-trained LLMs for my retrieval-augmented generation tool, focusing on fine-tuning pre-existing knowledge for accuracy and contextual relevance rather than building a model from scratch. This work, along with my investigation into vector embeddings and retrieval systems, emphasized the value of augmenting pre-trained models with context-specific data for dynamic query generation.

\subsection{RAG Literature}
After researching different methods to train LLMs, I learned about Retrieval Augmented Generation from Lewis et. al’s “Retrieval-Augmented Generation for Knowledge-Intensive NLP Tasks”.\cite{arxiv_2005_11401} The paper details how using a vector index filled with relevant data can enable language models to handle detail specific tasks more effectively by combining a query with relevant contextual information in the language model’s system prompt. The RAG framework also allows for dynamic updates that allow the model to be fed new information without needing to be trained. Although it is not implemented in my project, the ability to  efficiently update a language model is something to be explored in the future to enhance the long-term utility of such a tool. 
\vspace{25pt}
\subsection{Pinecone: Vector Databases and Embeddings}
To operationalize the RAG framework, I researched vector embeddings and databases, drawing on Roie Schwaber-Cohen’s Pinecone article, “What is a Vector Database and How Does it Work? Use Cases + Examples.”\cite{pinecone_vector_database} This resource emphasized how vector embeddings capture semantic meaning and facilitate similarity searches using metrics like cosine similarity and Euclidean distance. These insights informed the decision to use vector databases to store and retrieve semantic representations of Occidental College’s resources, enabling accurate and contextually relevant responses to user queries.

\section{Methods}
This section is dedicated to chronologically breaking down the process I took in creating my project and explaining the rationale behind my decisions.

\subsection{Data Collection}
To begin the project, I focused on gathering data related to Occidental College. To accomplish this I created a web crawling application using a combination of the Beautiful Soup (BS4) and Scrapy Python libraries. Tools like Selenium were considered but the combination of these libraries yielded very positive results. BS4 served to scrape the webpage content in plaintext while Scrapy systematically crawled through allowed domains to build a comprehensive file for Occidental’s online resources. For my allowed domain, I chose to only use resources found via the main Occidental College website as well as the Occidental Athletics site, the “oxy.edu” and “oxyathletics.com” domains to be specific. These domains were selected because they encompass the majority of resources that pertain to students, faculty and all interested parties like academic resources and athletic schedules. I found in scraping the Occidental website that there were some links that would create such a large error that it would halt the web crawling process early, so this required me to create a banned domains input in my program that checked against known troublesome domains. After compiling all of the text files, I noticed some files were just error messages, like “Whoops, no page found!” so they were systematically removed in order to reduce overall noise in the vector database. I also made the decision to remove all files under 20 characters in length, as they both added noise to the database and were extremely unlikely to contain meaningful information. The removal of these extraneous files makes it easier for the vector database to make its searches as the process weeded out around 300 files.

\subsection{Chunking and Sentence Transformers}
After the text files were collected and cleaned, my next step was to chunk the text data for efficient embedding and retrieval. I utilized the LangChain library for text chunking, in particular I used the recursive text splitter, which ensures that a chunk does not end in the middle of a word. Each page was divided into chunks of 500 characters with a 20-character overlap between each chunk to preserve semantic meaning across chunks. This chunk size was chosen to prioritize retrieval accuracy in the literature, which suggests that smaller chunks can better capture granular information while the overlap helps preserve context across chunks, ensuring that meaningful information is not lost between segments. By maintaining a balance between chunk size and overlap, the system can deliver precise and contextually relevant responses without overloading the database.
\newline
\indent For the vector database, I evaluated Pinecone and FAISS, two popular vector database options. While FAISS is known for its performance in offline environments, Pinecone was selected for its robust cloud-based capabilities, including scalability, support for real-time applications, and seamless integration with LangChain. Pinecone’s ability to handle dynamic updates, provide low-latency queries, and ease of use made it a better fit for this project’s needs, particularly given the expected user queries and the evolving nature of the Occidental College resources.
\newline
\indent The embeddings were generated using a SentenceTransformer model (all-mpnet-base-v2) with a vector size of 768 dimensions. This model was selected for its proven accuracy and efficiency in producing high-quality embeddings for semantic search tasks. The vector size of 768 strikes a balance between capturing complex relationships in the text and minimizing storage and computational overhead. I had previously tried a different sentence transformer, all-MiniLM-L6-v2. This sentence transformer had 384 dimensions during my testing. I found its results to not be as semantically complete as I would have liked. After generating the embeddings, each chunk was upserted into the Pinecone database, linking the embeddings with metadata such as the file title and chunk index to allow for precise and contextually aware retrieval during queries.
This approach ensured that the database was both comprehensive and efficient, enabling the retrieval-augmented generation (RAG) system to deliver accurate and relevant information to users in real-time. By combining LangChain’s chunking capabilities, Pinecone’s database management, and a carefully chosen embedding model, the project established a solid foundation for delivering high-quality query results.

\subsection{Natural Language Processing}
After building the foundational structure of my retrieval-augmented generation (RAG) system, I implemented natural language processing (NLP) techniques to refine the system’s performance and improve the relevance of retrieved results to the queries. These enhancements involved pre-processing queries, scoring titles based on query relevance, and leveraging NLP-based techniques to improve overall performance of my system. I would like to point out that stemming is not present in this project. This was due to time constraints, not because it was not a valuable form of NLP enhancement.

\subsubsection{Removing Stop Words}

\indent The first step in refining my system with NLP was removing stop words. Stop words are common terms such as "the," "and," or "is," which often add little value to semantic understanding in retrieval systems\cite{coursera_stop_words}. Stopwords are an issue because in a sentence like "The cat is brown", a sentence transformer will weight each word equally, meaning "the" and "cat" are seen as equally important, which can skew search results. Using NLTK's stop word dictionary, I filtered out these words from user queries before generating embeddings. This reduction allowed the model to focus on the meaningful words in the query, such as keywords or proper nouns, thus improving the alignment between the query and relevant documents. For instance, a query like "What are the financial aid opportunities?" was transformed into "financial aid opportunities," providing a much clearer definition for my RAG system to work with and retrieve useful results with.

\subsubsection{Title Scoring and Query Matching}

To further improve retrieval, I implemented a scoring mechanism that evaluated the degree of overlap between query words and words in document titles. Using a simple intersection-based approach\cite{kedia_methods_scoring_nlp}, I calculated how many terms from the processed query appeared in a document’s title. Titles with higher overlap received a boosted score. This adjustment was motivated by the assumption that titles are often concise summaries of a document's content, making them consistent indicators of relevance. For example, if a query was "internship opportunities at Occidental College," a document titled "Internship Programs and Resources" would score higher than one titled "Campus Dining Options."

\subsubsection{Impact of NLP Enhancements}

The combination of stop word removal and title scoring led to significant improvement to the retrieval accuracy of the RAG system. Queries began yielding results that were more semantically relevant to the user’s question and the system also prioritized documents with strong title-query relevance. This led to an increase in my system fetching relevant files, whereas before one or two of the three fetched files would tend to be  irrelevant to the query. For example, I tested the effectiveness of NLP using the query "Who is Jack Spencer", a friend of mine who runs on the Occidental Track and Field team. Without NLP, the only semi related file retrieved was one about actor and musician Jack Perry. With the NLP techniques implemented, the system was able to both find a file on Jack Spencer but also prioritize that file over less relevant files because his name was in the title.

The addition of these NLP techniques increased the ability of the RAG system to deliver precise, context-aware responses, underscoring the value of integrating semantic search with NLP-driven heuristics. These refinements reflect the system's ability to leverage linguistic insights to bridge the gap between complex user queries and the dense repository of Occidental College’s online resources.

\subsection{Testing Different Large Language Models}
To evaluate the importance of different language models for my retrieval-augmented generation (RAG) system, I tested four language models: GPT-2, Mistral 7B, GPT-3.5-turbo, and GPT-4o. Each model was integrated into the RAG system, including NLP enhancements, and assessed for its ability to accurately respond with relevant information based on the user queries related to the Occidental College dataset. This testing process aimed to identify the model that could balance accuracy and relevance while handling varied user inputs.\newline
\indent GPT-2 was selected as the baseline model, due to its significance as one of the earlier attempts at creating a large language model\cite{radford2019gpt2}. Its limitations quickly became apparent as the model frequently produced off-topic responses, even to straightforward queries such as “Tell me about Occidental College.” Despite the system feeding relevant context to the model, GPT-2 struggled to utilize it highlighting its inadequacy for the task.\newline
\indent Mistral 7B \cite{jiang2023mistral7b}, a free open-source model, demonstrated noticeable improvements over GPT-2. While it was better at producing contextually relevant responses, the relevancy of its output was dependent on users inputting highly specific queries, making it less useful to user queries with open-ended or vague context. This limitation underscored the importance of choosing a model capable of generalizing well across a variety of question styles.\newline
\indent The most successful models were GPT-3.5 \cite{ye2023gpt3_analysis} and GPT-4o\cite{openai2024gpt4technicalreport}, both of which consistently produced accurate and detailed responses. These models excelled in leveraging the provided context, requiring less precise input phrasing to generate high-quality answers. Their ability to interpret queries with minimal context and extract relevant information made them the best fit for the RAG system’s goal of providing accurate and reliable information to Occidental College users.
The choice to compare these specific models was informed by the need to balance computational efficiency with performance. GPT-2 served as a baseline, while Mistral 7B represented a smaller, more computationally efficient option. GPT-3.5 and GPT-4o, as state-of-the-art models, provided a benchmark for cutting-edge performance. This multi-model comparison ensured that the final system design was based on empirical evidence rather than assumptions. In the end, I chose to stick with GPT-4o, as it was the most up-to-date of all the LLMs and produced consistent results.


\section{Evaluation Metrics}
To evaluate the responses generated by retrieval-augmented generation (RAG) system, I developed a comprehensive set of evaluation metrics based on a range of question difficulties: Easy, Medium, and Hard. Easy questions required the system to provide straightforward factual information, such as “What is the mascot of Occidental College?” Medium questions involved a basic level of analysis or synthesis, such as “What are housing options at Occidental College?” Hard questions required deeper analysis and integration of information from multiple sources, such as “What are the benefits of the liberal arts education that Occidental College offers?” For each question, I documented the expected response and compared it to the system’s actual output to determine its accuracy and relevance\cite{finetunedb_evaluate_llm_outputs}. Expected answers were found by researching what the Occidental website provided in response, as well as things that would generally make sense to be present for the given query. Note that the expected responses I am evaluation against are a list of potential key ideas, so not every idea I list needs to be in the response to be considered relevant. I structured the evaluation this way as I need to set a 500 token limit to my model, so responses are not likely to capture every idea. This methodology ensured a structured and consistent approach to evaluating the RAG system's ability to process and retrieve information across different complexities. By using a balanced dataset of five questions for each difficulty level, I was able to thoroughly test the system’s robustness and highlight areas for future improvement. Here is a list of the questions I used along with the expected results:

\subsection{Easy Questions}
\begin{itemize}
    \item E1: Who is the President of Occidental College?
    Expected Answer: Tom Stritikus
    \item E2: What are the school colors of Occidental College?
    Expected: Orange, Black
    \item E3: What is the Occidental College Mascot
    Expected Answer: Tiger, Oswald
    \item E4: What sports are offered at Occidental College
    Expected Answer: Basketball, Soccer, Swimming, Track and Field, etc
    \item E5: What year was Occidental College founded?
    Expected Answer: 1887
\end{itemize}

\subsection{Medium Questions}
\begin{itemize}
    \item M1: What are housing options at Occidental College
    \newline Expected Answer: On campus halls (Berkus, Pauley, etc), Off campus options, options for faculty
    \item M2: How can students apply for an on-campus job?
    \newline Expected Answer: Job Boards, work study, types of jobs 
    \item M3: What are meal plan options for students?\newline
    Expected Answer: Meal plan A-c for First years, meal plan D for Sophmores, Juniors, and Seniors
    \item M4: What are the requirements for joining the choir group?\newline
    Expected Answer: Audition, comments about the application process, Glee Club
    \item M5: What is the process for reserving a study room in the library?\newline
    Expected Answer: visit library website, review the rules to reserve, select a time, Master Calendaer
\end{itemize}

\subsection{Hard Questions}
\begin{itemize}
    \item H1: What are the benefits of the liberal arts education that Occidental College offers?\newline
    Expected Answer: Critical thinking, problem solving, Internship / Research Opportunities
    \item H2: What grocery stores options are near Occidental Colege\newline
    Expected Answer: The cooler, The MP [Note: I included this question as it was something a student testing my system asked. I thought it captured a complex question that a real student would ask]
    \item H3: What are some interdisciplinary programs at Occidental College, and what are their key features?\newline
    Expected Answer: DWA, Core program, Cognitive Science, Critical Theory Social Justice
    \item H4: How does Occidental College support student-athletes academically? \newline
    Expected Answer: Audition, comments about the application process, Glee Club
    \item H5: What financial aid options are available at Occidental College, and how can students determine which ones they qualify for?
    \newline
    Expected Answer: visit library website, the rules to reserve, select a time, Master Calender
\end{itemize}


\section{Results and Discussion}
This section will serve to rate and evaluate the responses for each question. To keep this section concise I will only quoting the most relevant parts of the responses. To see the full responses, please refer to the appendices section of the paper. 

\subsection{Easy Questions}
The system demonstrated excellent performance with the easy-level questions, meeting or exceeding expectations in most cases. However, there was one minor exception with the question, "What are the school colors?". The response indicated that no resource in the database explicitly contained this information, but it guided the user toward an alternative method of discovery: examining the school logo or merchandise on Occidental’s website. This proactive approach showcased the system’s ability to compensate for data gaps by providing practical advice, which aligns with the goal of providing helpful guidance even in the absence of direct answers.

\begin{itemize}
\item: E1: "As of the latest information available, the President of Occidental College is Tom Stritikus."
\item: E3: "Occidental College's mascot is the tiger."
\item: E4: "At the varsity level, Oxy boasts 20 teams that compete in various sports. This robust lineup includes traditional team sports such as basketball, soccer, and baseball, which are staples in collegiate athletics."
\item: E5: "Occidental College, often referred to as Oxy, has a rich history that spans over a century. Its founding in 1887... "
\end{itemize}

\subsection{Medium Questions}
The RAG system performed consistently well on medium-level questions, blending information from multiple sources to deliver accurate, contextually rich responses. For M1 (Housing Options), the system successfully outlined housing options for both students and staff, including off-campus alternatives, demonstrating its ability to consolidate and present diverse resource details. For M2 (Work-Study Opportunities), the system provided a detailed explanation of work-study hiring priorities, highlighted the use of online job boards, and noted that certain roles, such as those related to MAC labs, require specific skillsets like proficiency in Adobe software. This response effectively combined technical and practical information, making it highly useful for the user. In M3 (Meal Plans), the system not only listed all available meal plans but also correctly identified a unique condition: Meal Plan D becomes accessible only after a student’s first year. This demonstrates the system’s capacity for extracting nuanced information from the dataset. For M4 (College Choir Application Process), the system outlined the steps involved in applying for the college choir and added value by mentioning glee club as an alternative, drawing a meaningful connection between the two groups. This highlights the system’s ability to provide comprehensive answers and identify relevant related information. Finally, for M5 (Library Room Reservations), although the system noted that no explicit guidelines for reserving a library room were available, it inferred that the process might be less formal than using the Master Calendar, which is typically required for larger events like club meetings. This reasoning demonstrated the system’s capability to fill in gaps with logical inferences while remaining grounded in the provided data. These results illustrate that the system excelled at medium-level questions by leveraging its contextual understanding and retrieval capabilities. However, the occasional reliance on inferences, particularly in M5, suggests potential areas for improvement in the richness and diversity of the dataset. Overall, the system effectively met the goals of accuracy and usability in this category, aligning with the broader objectives of the project.
\begin{itemize}
\item: M1: "*On-Campus Housing for Students:*
Occidental College mandates that students live on campus for their first three years...*Off-Campus Housing for Students:*...*Rental Housing for Faculty and Staff:*..."
\item: M2: "1. *Understanding Eligibility and Job Availability*:
- Students with a Federal Work-Study or Occidental Work Award have priority for on-campus jobs..."
\item: M3: "For students in their second year and beyond, as well as those living off-campus, there is an additional meal plan option known as Meal Plan D."
\item: M4: "Firstly, to join either the College Chorus or the Glee Club, students must participate in an audition process. This is a standard requirement to ensure that participants have the necessary vocal skills and commitment to contribute positively to the ensemble."
\item: M5: "For any space reservation, including study rooms, it is crucial to utilize the Master Calendar registration and approval process. "
\end{itemize}

\subsection{Hard Questions}
The hard-level questions posed the greatest challenge for the RAG system. Overall, the system produced mixed results, excelling in some areas while revealing gaps in others. For H1, the response effectively highlighted the importance of developing relevant skills and emphasized the unique opportunities available due to Occidental’s proximity to prestigious institutions like Caltech and UCLA. This demonstrated the system’s ability to integrate contextual details about Occidental’s location to provide a meaningful and comprehensive answer. H2 was particularly notable as it was one of the few questions sourced from a first-time user of the system. Despite the lack of explicit information on grocery stores in the Occidental dataset, the system ingeniously used Occidental’s address to enable the LLM to locate nearby stores via its external resources. This showcases the potential of the system to address user needs even when the dataset is incomplete, though reliance on LLM-generated external knowledge might introduce accuracy concerns. For H3, the system offered a strong response, emphasizing Occidental’s commitment to interdisciplinary education. It mentioned how the Core Program encourages students to explore diverse class selections, effectively addressing the question while reinforcing the institution’s values. H4 was the least effective, as the system failed to retrieve any specific resources to support its answer. The response relied heavily on vague language such as “likely” and “may,” reflecting a lack of usable data for this query. This highlights a key limitation in the dataset and retrieval process when dealing with less-defined or under-documented topics. For H5, the system provided a clear and well-structured answer, detailing Need-Based Aid, Merit-Based Scholarships, and Work Study Opportunities. It effectively summarized the types of aid available and the requirements for eligibility, offering a practical and actionable response. While the system demonstrated strengths in synthesizing complex information for some hard questions, it struggled when the dataset lacked depth or specificity. These results suggest that enhancing the diversity and granularity of the dataset could improve the system’s ability to address high-level queries with precision and confidence.


\begin{itemize}
\item: H1: "One of the primary benefits of a liberal arts education at Occidental is the development of essential skills such as critical thinking, writing, and communication"
\item: H2: "One of the closest grocery stores to Occidental College is Trader Joe's, a popular choice among students for its affordable prices and unique product offerings"
\item: H3: "A notable example of Occidental's interdisciplinary offerings is the Core Program, which is designed to encourage free exploration of academic subjects."
\item: H4: "Firstly, Occidental College likely provides academic advising tailored to the unique needs of student-athletes."
\item: H5: "The financial aid options available at Occidental College include need-based financial aid, merit-based scholarships, work-study opportunities, and student loans. "
\end{itemize}

\subsection{Evaluation}
\indent The primary goal of this project was to create a RAG system capable of addressing the difficulties users face when navigating Occidental College’s dense and complex website. The system aimed to provide accurate and centralized answers to user queries by leveraging a LLM with a curated dataset of Occidental’s resources. Based on the results from testing across easy, medium, and hard-level questions, I believe the project achieved its primary objectives, though there is definitely room for improvement in addressing complex queries. For easy-level questions, the system demonstrated near-perfect accuracy, efficiently retrieving and synthesizing relevant information for straightforward queries. Even when the exact answer was unavailable, such as identifying the school colors, the system provided reasonable guidance, directing users to alternative resources. This highlights the system’s effectiveness in handling factual queries and offering practical solutions when direct answers were missing.

\indent The system also performed adequately on medium-level questions, consistently providing accurate and contextually rich responses. It successfully addressed questions about housing options, applying for on-campus jobs, and meal plans all while including nuanced details, such as the availability of specific meal plans based on student year. Additionally, the system displayed its ability to connect related resources, as seen in its response about the college choir and glee club, further reinforcing the effectiveness of the RAG approach.

\indent The hard-level questions posed the greatest challenge to the viability of my system. While the system excelled in some areas, such as highlighting interdisciplinary education and leveraging Occidental’s address to answer an external query, it struggled with less-defined topics. The vague and speculative response to student-athlete academic support underscores the limitations of the current dataset and retrieval process in addressing high-level queries requiring greater detail and precision.

\indent Overall, the RAG system succeeded in addressing the core problem of simplifying access to Occidental College’s resources. The combination of accurate responses, practical guidance, and the ability to blend related information demonstrates the system’s value as a tool for improving user navigation. However, the limitations identified in handling highly complex or ambiguous queries suggest that further refinement, particularly in expanding and enriching the dataset, would enhance the system’s robustness and reliability. Despite these challenges, I am confident that the project has made significant strides in solving the problem and has laid a strong foundation for future iterations.

\section{Ethical Consideration}
One of the most significant ethical considerations for my project is hallucinations, where a LLM generates an incorrect or misleading response\cite{lakera_hallucinations_llms}. Although accuracy was a top priority when developing this project, hallucinations remain a potential risk, especially when the system encounters queries with no supporting files like the student-athlete query. Another ethical concern involves the inclusion of personal information\cite{roy2024protecting}. While I deliberately avoided scraping sensitive data, such as grades or private details, the athletics website includes publicly available information about student-athletes, and their consent was not obtained for inclusion in the dataset. There is also a  risk of misrepresentation or bias in responses, as the system relies on the Occidental College to provide neutral, unbiased coverage for all of its information\cite{orange_llm_bias}. Any gaps or skewed perspectives in the data could inadvertently spread biases, potentially enforcing stereotypes against certain groups or misinforming users. Bias is a reason I chose to not include the Occidental newspaper in my dataset, as news tends to contain bias, harmuful or not.\cite{lwv_media_bias}. Furthermore, the project currently lacks robust security measures to protect the data from misuse or breaches, raising concerns about safeguarding the dataset and ensuring compliance with data protection standards\cite{kumar2024ethicsinteractionmitigatingsecurity}.

\section{Replication Instruction}
In order to recreate this project, it is important that Python, Anaconda, and all other necessary libraries are installed on you computer.
Here is a list of all libraries used. Note that these are the libraries needed to run the final system I made.
\begin{itemize}
    \item Scrapy 2.11.2
    \item bs4 0.0.2 
    \item Langchain 0.3.0
    \item OpenAI 1.52.2
    \item nltk 3.9.1
    \item pinecone 5.1.0
    \item sentence-transformers 3.2.1
\end{itemize}
It is also important that you have API keys for both OpenAI and Pinecone to be able to access the LLMs as well as the vector database. To get the vector database information to access it, please email me at zchoudhry@oxy.edu. Other than this, make sure to clone my repository and use the file "DemoGPT4o.py" to use the same system I did. Any other libraries seen in the code not mentioned in this section come with Python / Anaconda when installed


\section{Code Architecture Review}
This section will serve both as a Code Architecture Review as well as an appendices. My code will be broken down into 3 sections, Retrieval, Database Upsertion, and Response Generation. After this, I will post my full prompts along with their full responses. Please note for this first section that the only files mentioned will be the ones I used to contribute to the final system. All other files are not relevant to this end.
\subsection{Retrieval}
\begin{itemize}
    \item crawling-tiger.py: located in the "Oxycrawler" folder, inside the "spiders" folder. This is where all the webscraping and webcrawling is handled. Feel free to edit the allowed / banned domains as well as tamper with the custom setting for your scraping needs.
\end{itemize}

\subsection{Database Upserting}
\begin{itemize}
    \item Pinecone-chunker.py: This file is located in the LLM folder. After the data is scraped, run this file to chunk, embed and upload all the files into the database. In this file, you can change chunk and overlap size, as well as try out different sentence transformers / vector dimensions to suit your needs. Make sure you set the folder that you want the program to pull its files from.
\end{itemize}

\subsection{Response Generation}
\begin{itemize}
    \item Demo-GPT4o.py: located in the LLM folder. This file is where users can re-create the results from my experiment. Present in the file is all of the NLP techniques listed prior, as well as the system prompt. If you do change the sentence transformer in "Pinecone-chunker.py" ensure the sentence transformer is this file matches.
\end{itemize}

\subsection{Full Code Responses}
Please note that some of the responses cut off mid-sentence. This is due to the hard cap of 500 tokens that I put on the LLM.

\newline
\subsubsection{Easy Questions}
\begin{itemize}
\item Who is the President of Occidental College? \newline
\indent As of the latest information available, the President of Occidental College is Tom Stritikus. He is the 17th president of the institution. In his introductory message, President Stritikus expressed his excitement and gratitude for being selected to lead Occidental College. He shared his reflections on the vibrant and engaged community of students that make up the college, highlighting his anticipation for the fall when students return to campus. This message underscores his commitment to fostering a dynamic and service-oriented educational environment at Occidental.
\newline
\indent President Stritikus's leadership comes at a time when Occidental College is actively pursuing significant initiatives, such as The Oxy Campaign For Good. This campaign aims to raise 225 million to support various aspects of the college, including new student scholarships and endowed professorships. Despite the challenges posed by the pandemic, the college has successfully raised more than two-thirds of this goal, demonstrating the resilience and dedication of its community under President Stritikus's guidance.
\newline
\indent His presidency is marked by a focus on maintaining Occidental's status as a national leader in liberal arts education. The college's commitment to providing a transformative educational experience is evident in its efforts to engage students in meaningful ways, both on and off campus. President Stritikus's leadership is characterized by a vision of Occidental as an institution that not only educates but also inspires students to become active and thoughtful contributors to society.
\indent Furthermore, President Stritikus's tenure at Occidental College is part of a broader narrative of leadership transitions at the institution. His appointment follows a decade-long presidency, as noted in a letter marking the 10-year anniversary of his predecessor's tenure. This continuity in leadership reflects Occidental's dedication to stability and progress in its mission to deliver high-quality liberal arts education.
\newline
\item What are the school colors of Occidental College? \newline
\indent The documents provided do not explicitly mention the school colors of Occidental College. However, I can provide some context and guidance on where you might typically find such information and why it might be important. School colors are a significant part of a college's identity and branding. They are often used in the college's logo, merchandise, athletic uniforms, and promotional materials. For Occidental College, like any other institution, the school colors would be a key component of its visual identity, helping to create a sense of unity and pride among students, faculty, alumni, and supporters. While the documents do not specify the colors, they do emphasize the importance of consistent branding and the use of the Occidental College logo in maintaining the college's identity. The logo, as described, is a visual shorthand representing the college's values and community. It is likely that the school colors are incorporated into this logo and other branding materials. To find the specific school colors, you would typically look at the college's official website, particularly sections related to branding or marketing communications. The Office of Marketing and Communications at Occidental College, as mentioned in the documents, would be a valuable resource for this information. They provide a style guide and templates to ensure that all materials are consistent with the college's branding, which would include the correct use of school colors. Additionally, school colors are often prominently displayed during college events, on athletic uniforms, and in campus facilities. They are a source of pride and are used to rally support during sports events and other college activities. If you are interested in learning more about the school colors of Occidental College, I recommend visiting the college's official website or contacting the Office of Marketing and Communications directly. They would be able to provide the most accurate and up-to-date information regarding the college's branding and visual identity, including the school colors. In summary, while the specific school colors of Occidental College are not mentioned in the provided documents, they are an integral part of the college's branding and identity. To find this information, you would typically consult the college's official resources, such as the website or the Office of Marketing and Communications. These colors play a crucial role in representing the college's spirit and values across various platforms and events.
\newline
\item What is the Occidental College Mascot
Occidental College's mascot is the tiger. This is evident from the document titled news-occidental-college-collaborates-community-partners-yosemite-drive-street-mural.txt, which mentions that the design of a street mural includes tiger paw prints as a nod to Occidental’s tiger mascot. The tiger serves as a symbol of the college's spirit and identity, representing the institution in various athletic and community events.The choice of a tiger as a mascot is significant for several reasons. Tigers are often associated with qualities such as strength, courage, and tenacity, which are attributes that many educational institutions aspire to instill in their students. By adopting the tiger as its mascot, Occidental College aligns itself with these values, promoting a sense of pride and unity among its students, faculty, and alumni.
Mascots play an important role in college culture, serving as a rallying point for school spirit and a symbol of the institution's identity. They are often featured in college branding, merchandise, and at athletic events, where they help to engage the community and foster a sense of belonging. At Occidental College, the tiger mascot likely serves as a unifying figure, bringing together the college community in support of its teams and initiatives. In addition to its role in athletics, the tiger mascot may also be used in various college communications and marketing materials. As indicated in the policy-directory-use-college-name-and-trademarks-policy.txt, the use of college marks, including mascots, is carefully regulated to ensure consistency and protect the college's brand identity. This policy underscores the importance of the tiger mascot as a key element of Occidental College's visual and cultural representation. Overall, the tiger mascot is an integral part of Occidental College's identity, symbolizing the institution's values and fostering a sense of community among its members.
\newline
\item What sports are offered at Occidental College
\indent Occidental College, commonly referred to as Oxy, offers a vibrant and comprehensive athletics program that is an integral part of the student experience. The college provides a wide array of sports opportunities that cater to a diverse range of interests and skill levels, ensuring that nearly half of the student body is actively engaged in some form of athletic activity. At the varsity level, Oxy boasts 20 teams that compete in various sports. This robust lineup includes traditional team sports such as basketball, soccer, and baseball, which are staples in collegiate athletics. Additionally, Oxy offers sports like track and field, swimming, and tennis, which provide students with opportunities to excel in individual events while contributing to a team dynamic. The college's commitment to a well-rounded athletic program is evident in its support for both team and individual sports, allowing students to pursue their athletic passions in a competitive environment. Beyond varsity athletics, Occidental College also supports several club sports. These clubs provide a platform for students who wish to engage in competitive sports without the commitment required at the varsity level. Club sports often include activities such as rugby, ultimate frisbee, and cycling, among others. These teams typically compete against clubs from other colleges and universities, offering a competitive yet more flexible alternative to varsity sports. In addition to varsity and club sports, Oxy has a thriving intramural program that encourages participation from the entire student body. Intramural sports are designed to be inclusive and fun, allowing students of all skill levels to participate in a variety of activities. Common intramural sports include flag football, basketball, and volleyball, providing students with opportunities to stay active and engage in friendly competition with their peers.
\newline
\item What year was Occidental College founded?
\newline
Occidental College, often referred to as Oxy, has a rich history that spans over a century. Its founding in 1887 marked the beginning of an institution that would grow to become a significant part of the educational landscape in Southern California. The College was established with a mission to provide a comprehensive liberal arts education, and over the years, it has maintained its commitment to academic excellence and fostering a diverse and inclusive community. The founding of Occidental College in the late 19th century was part of a broader movement of establishing higher education institutions across the United States. During this period, there was a growing recognition of the importance of higher education in advancing society and preparing individuals for the challenges of the modern world. Occidental's establishment was a response to this need, and it quickly became known for its rigorous academic programs and its emphasis on critical thinking and intellectual engagement. Throughout its history, Occidental has faced various challenges and opportunities, adapting to the changing educational landscape while remaining true to its core values. The College's commitment to diversity, equity, and inclusion has been a defining characteristic, and it has played a significant role in shaping the institution's identity and mission. Occidental's founding year of 1887 is not just a historical fact but a testament to the enduring legacy of the College. It serves as a reminder of the vision and dedication of its founders, who sought to create an institution that would stand the test of time and continue to make a positive impact on the world. As Occidental continues to evolve and grow, its founding year remains a cornerstone of its identity, reflecting the College's rich history and its ongoing commitment to excellence in education.
\end{itemize}

\subsubsection{Medium Questions}
\begin{itemize}
\item What are housing options at Occidental College
Occidental College offers a variety of housing options tailored to meet the diverse needs of its student body, faculty, and staff. The housing landscape at Occidental is designed to foster a vibrant community while accommodating different preferences and requirements.*On-Campus Housing for Students:*Occidental College mandates that students live on campus for their first three years. This policy ensures that students are integrated into the campus community, benefiting from the social and academic opportunities that on-campus living provides. The majority of students reside in double or triple rooms, as single rooms are not commonly available. This arrangement encourages students to engage with their peers and build lasting relationships.
For student military veterans, Occidental offers a unique housing option at the off-campus Toland Way housing complex. This facility provides apartment-style living, catering specifically to the needs of veterans who may require a different living arrangement than traditional dormitory-style housing.*Off-Campus Housing for Students:*
While the college requires underclassmen to live on campus, juniors have the opportunity to apply for an exception to live off campus. Those who do typically reside in off-campus Greek houses or houses with a tradition of housing student athletes. It's important to note that these off-campus properties are not managed by Occidental College, meaning students must navigate leases and housing agreements independently.
For seniors, about 50 percent choose to remain on campus, while the other half opt for off-campus living. This flexibility allows seniors to experience living independently while still being closely connected to the college community.*Rental Housing for Faculty and Staff:*Occidental College extends its housing offerings beyond students to include faculty, administrators, and staff through its College-owned rental housing program.
\newline
\item How can Students apply for on-campus jobs?
Applying for campus jobs at Occidental College involves several steps and considerations, especially for students who are part of the work-study program and those who are not. Here’s a comprehensive guide on how students can navigate the process of securing on-campus employment:1. *Understanding Eligibility and Job Availability*: - Students with a Federal Work-Study or Occidental Work Award have priority for on-campus jobs. However, there are opportunities for students without these awards as well. Certain departments are open to hiring non-work-study students, and these positions are clearly marked in job descriptions. It’s important for students to check the eligibility requirements for each job posting to ensure they meet the criteria.
2. *Finding Job Postings*:- Current job postings are available online, and students should regularly check these listings to find opportunities that match their skills and interests. The Office of Student Employment is responsible for processing all student employment, and they provide resources and support for students seeking jobs.3. *Application Process*:- Students interested in applying for campus jobs need to complete the Student Employment Application. It’s crucial for MAC users to open this application in Adobe, as using Preview might result in data not being saved. Some departments may also require a supplemental application, so students should be prepared to provide additional information if needed.4. *Required Documentation*:- To work on campus, students must complete the I-9 and W-4 forms, which require original approved government-issued identification. It’s important to note that copies of these documents will not be accepted, so students should ensure they bring the necessary original documents to campus.5. *Types of Positions Available*:- The college offers a variety of student positions, such as Student Compost Facilitator,
\newline
\item What are the meal plan options for students?
Occidental College offers a variety of meal plan options to accommodate the diverse dining needs and preferences of its students. The meal plans are designed to provide flexibility and convenience, ensuring that students have access to nutritious meals throughout their time on campus. The meal plan options available to students are determined by their class year and housing situation, with specific plans tailored for first-year students and upperclassmen.For first-year students, the meal plan options are somewhat limited to ensure that they have ample access to on-campus dining facilities as they acclimate to college life. The available meal plans for first-year students living on-campus include Meal Plan A+, Meal Plan A, Meal Plan B, and Meal Plan C. These plans are structured to provide a comprehensive dining experience, with varying levels of meal credits and flexibility.
Meal Plan A+ is the most comprehensive option, priced at $9,322 for the academic year, or $4,661 per semester. This plan is ideal for students who prefer to have the most dining flexibility and access to a wide range of dining options on campus. Meal Plan A, slightly less comprehensive than A+, is priced at $8,652 annually, or $4,326 per semester. It offers a robust dining experience with a significant amount of meal credits.Meal Plan B, priced at $8,294 annually or $4,147 per semester, provides a balanced option for students who want a moderate level of dining flexibility. Meal Plan C, the most economical option available to first-year students, costs $7,534 annually or $3,767 per semester. This plan is suitable for students who may have lighter dining needs or prefer to supplement their on-campus meals with other options.
For students in their second year and beyond, as well as those living off-campus, there is an additional meal plan option known as Meal Plan D. This plan is not available to first-year students and is priced at $6,888 annually or $3,444 per semester. Meal Plan D offers a more budget-friendly option for upperclassmen
\newline
\item What are the requirements for joining the choir group?
Joining a choir group at Occidental College involves several steps and requirements, particularly if you are interested in the College Chorus or the Glee Club, which are foundational ensembles within the college's music program. Firstly, to join either the College Chorus or the Glee Club, students must participate in an audition process. This is a standard requirement to ensure that participants have the necessary vocal skills and commitment to contribute positively to the ensemble. For the Fall 2020 semester, auditions were conducted online, reflecting adaptations to circumstances such as the COVID-19 pandemic. Prospective members were required to fill out a form to initiate the audition process. While this specific detail pertains to Fall 2020, it indicates the college's flexibility and adaptation to current events, suggesting that audition formats may vary depending on circumstances. Membership in either the College Chorus or the Glee Club is also a prerequisite for auditioning for any A Cappella group at Occidental College. This requirement underscores the importance of these foundational ensembles in the college's vocal music ecosystem, serving as a stepping stone for more specialized vocal groups. The College Chorus, designated as MUSC 120, is a 1-unit course where singers engage with a diverse repertoire, including classical, world music, musical theater, and pop. The ensemble rehearses once a week on Monday evenings from 7:15 to 9:45 PM and performs on campus once each semester. Enrollment in the College Chorus can be for one or both semesters, providing flexibility for students to manage their academic and extracurricular commitments. For the Glee Club, there are two specific groups based on vocal range: the Glee Club for Sopranos/Altos (MUSC 121) and the Glee Club for Tenors/Basses (MUSC 122). Each of these courses offers 0 or 2 units, indicating that participation can be tailored to the student's academic plan and credit requirements.
\newline
\item What is the process for reserving a study room in the library?
Reserving a study room in the library at Occidental College involves a structured process that ensures compliance with the College's policies and efficient use of resources. While the documents provided do not explicitly detail the process for reserving a study room specifically within the library, they do outline the general procedure for scheduling events and spaces on campus, which can be extrapolated to include study room reservations. The first step in reserving a study room would be to consult the Event Planning Resources web page. This page is a comprehensive resource that provides essential information about planning and scheduling events in accordance with both College and legal requirements. Although primarily aimed at larger events, the principles and guidelines outlined can be beneficial for understanding the general framework for space reservations. For any space reservation, including study rooms, it is crucial to utilize the Master Calendar registration and approval process. This process is mandatory for all Category 2-7 activities, which likely includes study room reservations, given their structured nature within the campus facilities. The Master Calendar system is designed to streamline the scheduling process and ensure that all necessary approvals are obtained. To initiate a reservation, you would typically need to submit a request through the Master Calendar system. This system is the central hub for scheduling and is used to manage all space reservations on campus. The request process involves providing details about the intended use of the space, the number of participants, and any specific requirements or resources needed. Once a request is submitted, it may require approval from relevant College authorities. For larger events, this might involve review by the Campus Events Advisory Committee (CEAC) or other administrative offices. However, for a study room reservation, the approval process is likely more straightforward, possibly involving only the library staff or a designated space
\newline
\end{itemize}

\subsubsection{Hard Questions}
\begin{itemize}
\item What are the benefits of the liberal arts education that Occidental College offers?
Occidental College, located in Los Angeles, offers a liberal arts education that is both comprehensive and distinctive, providing students with numerous benefits that prepare them for a wide range of future endeavors. The college's approach to liberal arts education is designed to hone critical skills, foster intellectual growth, and provide unique opportunities that are enhanced by its strategic location.

One of the primary benefits of a liberal arts education at Occidental is the development of essential skills such as critical thinking, writing, and communication. These skills are cultivated in an intellectually rigorous and collaborative environment, ensuring that students are well-prepared to tackle complex problems and communicate effectively in various contexts. The emphasis on these skills is crucial, as they are highly valued in today's rapidly changing job market and are applicable across numerous fields and industries.

Occidental's reputation as an elite liberal arts college is bolstered by its strategic location in Los Angeles, a vibrant city that offers a wealth of opportunities for students. The college's proximity to top research institutions such as Caltech, UCLA, USC, the Huntington Library, and the Getty Research Institute provides students with easy access to cutting-edge research and academic resources. This access allows students to engage in meaningful research projects, internships, and collaborations that enhance their academic experience and prepare them for future careers or advanced studies.

Moreover, the location in Los Angeles offers students the chance to experience a small college environment while benefiting from the cultural, professional, and social opportunities of a large city. This unique combination attracts students who are interested in a personalized educational experience without sacrificing the advantages of a metropolitan setting. The city's diverse industries, including entertainment, technology, and finance, provide a rich landscape for internships and job opportun

\item What are grocery store options near Occidental College?

Occidental College, located in the vibrant Eagle Rock neighborhood of Los Angeles, offers its students a variety of nearby grocery store options to meet their diverse needs. While the college itself provides several on-campus dining options, such as the Marketplace, Tiger Cooler, Green Bean Coffee Lounge, and the Coffee Cart, students often seek out local grocery stores for additional food supplies, personal items, or specialty products. One of the closest grocery stores to Occidental College is Trader Joe's, a popular choice among students for its affordable prices and unique product offerings. Trader Joe's is known for its wide selection of organic and specialty foods, making it a great option for students who are health-conscious or have specific dietary preferences. The store's proximity to campus makes it a convenient choice for quick grocery runs. Another nearby option is Vons, a well-known supermarket chain that offers a comprehensive range of products, including fresh produce, meats, dairy, and household items. Vons is a convenient choice for students looking to do a larger grocery shop or those who need to pick up a variety of items in one trip. The store often has sales and a loyalty program, which can be beneficial for budget-conscious students. For those interested in organic and natural foods, Sprouts Farmers Market is another excellent option. Sprouts is renowned for its fresh produce, bulk foods, and health products. It caters to students who prioritize organic and natural food options and offers a wide range of vitamins and supplements, which can be appealing to those focused on health and wellness. Additionally, there is a Ralphs supermarket in the vicinity, providing another comprehensive grocery shopping experience. Ralphs offers a wide selection of everyday essentials, fresh produce, and ready-to-eat meals, which can be particularly convenient for busy students. The store's loyalty program can also help students save money on their purchases.

\item What are some interdisciplinary programs at Occidental College, and what are their key features? Occidental College, a liberal arts institution located in Los Angeles, is renowned for its commitment to interdisciplinary education, which is a cornerstone of its academic philosophy. This approach is designed to prepare students for leadership in a complex, interdependent, and pluralistic world, as highlighted in the college's mission statement. The interdisciplinary programs at Occidental College are characterized by their integration of diverse fields of study, fostering a holistic educational experience that encourages critical thinking, creativity, and collaboration. One of the key features of Occidental's interdisciplinary programs is their focus on immersive and collaborative learning experiences. The college's dynamic faculty are deeply invested in interdisciplinary teaching, which often involves collaborative ventures that transcend traditional departmental boundaries. This approach allows students to engage with complex issues from multiple perspectives, enhancing their ability to think critically and solve problems creatively. A notable example of Occidental's interdisciplinary offerings is the Core Program, which is designed to encourage free exploration of academic subjects. This program provides a foundation for students to engage with a wide range of disciplines, promoting intellectual curiosity and flexibility. The Core Program is integral to the college's liberal arts education, as it ensures that all students, regardless of their major, have a shared academic experience that emphasizes the interconnectedness of knowledge.
Occidental also offers a variety of interdisciplinary majors and minors that allow students to tailor their education to their interests and career goals. These programs often combine elements from different departments, enabling students to explore the intersections of various fields. For instance, students might engage in studies that blend the humanities with the sciences, or the social sciences with the arts, to gain a comprehensive understanding of complex topics. 

\item How does Occidental College support student-athletes academically?

Occidental College is committed to supporting its student-athletes academically, recognizing the importance of balancing athletic commitments with academic responsibilities. This commitment is rooted in the college's broader mission to enhance the liberal arts student experience, which includes fostering both academic and athletic excellence. While the documents provided do not explicitly detail specific academic support programs for student-athletes, we can infer several ways in which Occidental College likely supports its student-athletes based on common practices in higher education and the college's emphasis on the holistic development of its students. Firstly, Occidental College likely provides academic advising tailored to the unique needs of student-athletes. Academic advisors who understand the demands of athletic schedules can help student-athletes plan their courses in a way that accommodates practices, competitions, and travel. This personalized advising ensures that student-athletes can meet their academic requirements without compromising their athletic commitments. 
Secondly, the college may offer tutoring services and academic workshops specifically for student-athletes. These resources can help student-athletes stay on top of their coursework and develop effective study habits. Tutoring sessions might be scheduled around athletic commitments, providing flexibility and support when student-athletes need it most. Additionally, Occidental College likely emphasizes the importance of time management skills for its student-athletes. Workshops or seminars on time management can equip student-athletes with strategies to balance their dual roles effectively. By learning how to prioritize tasks and manage their time efficiently, student-athletes can maintain their academic performance while fulfilling their athletic responsibilities.
The college's faculty and coaches also play a crucial role in supporting student-athletes academically. Faculty members who are aware of the challenges faced by student-athl

\item What financial aid options are available at Occidental College, and how can students determine which ones they qualify for?

Occidental College offers a comprehensive financial aid program designed to make education affordable for a diverse group of students. The financial aid options available at Occidental College include need-based financial aid, merit-based scholarships, work-study opportunities, and student loans. Understanding these options and determining eligibility is crucial for prospective and current students seeking financial assistance. *Need-Based Financial Aid:* Occidental College is committed to meeting 100 percent of the demonstrated financial need for all U.S. citizens, permanent residents, and undocumented students. This means the college will cover the gap between the total cost of attendance and what a family can afford to pay. To apply for need-based financial aid, students must submit the Free Application for Federal Student Aid (FAFSA) and the CSS Profile by the specified deadline. These applications assess a family's income and assets to determine the student's financial need.  For international students, Occidental College also offers need-based financial aid. International applicants must complete specific financial forms as part of their application process to be considered for this type of assistance. The college evaluates the financial circumstances of international students similarly to domestic students, ensuring that those with demonstrated need receive the necessary support.
*Merit-Based Scholarships:* Merit-based scholarships at Occidental College are awarded based on a student's academic achievements, extracurricular involvement, and personal accomplishments. These scholarships do not require a separate application; all applicants to the college are automatically considered based on the information provided in their admission application. Both domestic and international students are eligible for merit-based scholarships, which can significantly reduce the cost of attendance.
*Work-Study Opportunities:* Occidental College offers work-study programs that allow students to earn money to help cover th

\end{itemize}



\end{document}
